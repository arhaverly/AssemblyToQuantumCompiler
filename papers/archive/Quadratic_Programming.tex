Abstract:

In this paper, we present a novel approach for solving the Quadratic Programming (QP) problem using Grover's Algorithm, a quantum search algorithm with quadratic speedup over classical algorithms. QP is a fundamental optimization problem with wide-ranging applications in various fields such as finance, engineering, and machine learning. It involves optimizing a quadratic objective function subject to linear constraints. Although classical QP solvers have been developed and improved over the years, they still suffer from computational complexity limitations, especially for large-scale problems. Our proposed method leverages the power of quantum computing to find the optimal solution more efficiently, thus providing a promising alternative for addressing challenging QP problems. Additionally, we provide a detailed description of the developed algorithm, analyze its complexity, and discuss its potential applications.

\section{Introduction}

Quadratic Programming (QP) is an optimization problem that can be represented mathematically as follows:

\begin{equation}
\begin{aligned}
& \underset{x}{\text{minimize}}
& & \frac{1}{2}x^TQx + c^Tx \\
& \text{subject to}
& & Ax\leq b
\end{aligned}
\end{equation}

where $x \in \mathbb{R}^n$ is the decision variable vector, $Q \in \mathbb{R}^{n\times n}$ is a symmetric positive semi-definite matrix, $c \in \mathbb{R}^n$ is a linear term vector, $A \in \mathbb{R}^{m\times n}$ is a constraint matrix, and $b \in \mathbb{R}^m$ is a constraint vector. Quadratic programming problems have numerous applications in various fields, such as portfolio optimization in finance, support vector machines in machine learning, and control systems in engineering.

Classical algorithms for solving QP problems include interior-point methods, active-set methods, and gradient-based methods \cite{nocedal2006numerical}. These methods have been continuously improved over the years, but their performance is still limited by the computational complexity inherent in large-scale problems. With the advent of quantum computing, there is a growing interest in developing quantum algorithms that can outperform classical algorithms for various optimization problems, including QP.

Grover's Algorithm is a quantum search algorithm that can search an unsorted database of $N$ items in $\mathcal{O}(\sqrt{N})$ queries, providing a quadratic speedup over classical search algorithms \cite{grover1996fast}. This algorithm has been extended and adapted to solve combinatorial optimization problems, such as the traveling salesman problem and the graph coloring problem \cite{ambainis2000quantum}. In this paper, we propose a novel approach for solving the QP problem using Grover's Algorithm. We provide a detailed description of the developed algorithm, analyze its complexity, and discuss its potential applications.

The remainder of this paper is organized as follows. In Section \ref{sec:background}, we review the fundamentals of quantum computing, Grover's Algorithm, and the Quadratic Programming problem. In Section \ref{sec:algorithm}, we present the proposed quantum algorithm for solving the QP problem and provide a step-by-step description of its implementation. In Section \ref{sec:complexity}, we analyze the complexity of the proposed algorithm and compare it with classical QP solvers. In Section \ref{sec:applications}, we discuss potential applications of the proposed quantum QP solver in various fields. Finally, Section \ref{sec:conclusion} concludes the paper and outlines directions for future research.

\section{Background}
\label{sec:background}

In this section, we briefly review the fundamentals of quantum computing and Grover's Algorithm. Additionally, we provide an overview of the Quadratic Programming problem and its classical solution methods.

\subsection{Quantum Computing}

Quantum computing is a computational paradigm that exploits the principles of quantum mechanics to perform complex calculations more efficiently than classical computing. The fundamental unit of quantum information is the quantum bit or qubit, which can exist in a superposition of states $\ket{0}$ and $\ket{1}$, represented as $\ket{\psi} = \alpha\ket{0} + \beta\ket{1}$, where $\alpha$ and $\beta$ are complex coefficients such that $|\alpha|^2 + |\beta|^2 = 1$. Quantum operations are performed using quantum gates, which are unitary matrices that preserve the normalization of the qubit state vector. Some common quantum gates include the Pauli-X, Pauli-Y, Pauli-Z, Hadamard (H), and Controlled-NOT (CNOT) gates \cite{nielsen2010quantum}.

A quantum algorithm consists of a sequence of quantum operations applied to an initial state of qubits, followed by a measurement that collapses the qubits' state to a classical bit value. Quantum algorithms can exploit quantum parallelism, entanglement, and interference to solve problems more efficiently than classical algorithms.

\subsection{Grover's Algorithm}

Grover's Algorithm is a quantum search algorithm that can search an unsorted database of $N$ items with a quadratic speedup over classical algorithms \cite{grover1996fast}. The algorithm consists of two main steps: the oracle query and the Grover diffusion operator. The oracle query is a quantum operation that marks the solution state(s) by inverting their sign. The Grover diffusion operator amplifies the amplitude of the marked state(s) and reduces the amplitude of the non-marked states, effectively increasing the probability of measuring the solution state(s). After $\mathcal{O}(\sqrt{N})$ iterations of these two steps, the probability of measuring the solution state becomes close to 1.

\subsection{Quadratic Programming}

Quadratic Programming (QP) is a fundamental optimization problem that involves optimizing a quadratic objective function subject to linear constraints. QP problems can be convex or non-convex, depending on the properties of the quadratic objective function and the constraint set. In this paper, we focus on convex QP problems, which have a unique global minimum and can be solved efficiently using classical algorithms.

Classical algorithms for solving QP problems include interior-point methods, active-set methods, and gradient-based methods \cite{nocedal2006numerical}. Interior-point methods are based on the idea of barrier functions, which transform the constrained optimization problem into an unconstrained problem that can be solved iteratively. Active-set methods maintain a set of active constraints, iteratively updating the set and solving a sequence of equality-constrained QP subproblems. Gradient-based methods, such as the conjugate gradient method and the gradient projection method, exploit the gradient information of the objective function and the constraints to search for the optimal solution.

Despite the availability of various classical QP solvers, their performance is limited by the computational complexity of large-scale problems. Quantum computing offers a potential solution to this limitation by providing quantum algorithms that can outperform classical algorithms on certain optimization problems.

\section{Quantum Algorithm for Quadratic Programming}
\label{sec:algorithm}

In this section, we present the proposed quantum algorithm for solving the QP problem using Grover's Algorithm. The algorithm consists of the following steps:

1. Encode the QP problem into a quantum oracle that marks the optimal solution.

2. Initialize a quantum register with an equal superposition of all possible states using Hadamard gates.

3. Apply Grover's Algorithm, consisting of the oracle query and the Grover diffusion operator, for $\mathcal{O}(\sqrt{N})$ iterations.

4. Measure the quantum register to obtain the optimal solution.

\subsection{Encoding the QP Problem}

The first step of the algorithm is to encode the QP problem into a quantum oracle that can mark the optimal solution. This can be done by discretizing the decision variable space into a finite set of candidate solutions and designing a quantum circuit that computes the objective function and the constraints for each candidate solution. The oracle should be able to recognize the optimal solution based on the computed values and mark it by applying a phase shift.

\subsection{Initializing the Quantum Register}

The second step of the algorithm is to initialize a quantum register with an equal superposition of all possible states. This can be achieved by applying Hadamard gates to each qubit in the register. The initial state of the quantum register can be represented as

\begin{equation}
\ket{\psi_0} = \frac{1}{\sqrt{N}}\sum_{i=0}^{N-1}\ket{i},
\end{equation}

where $N$ is the number of candidate solutions and $\ket{i}$ is the $i$-th basis state.

\subsection{Applying Grover's Algorithm}

The third step of the algorithm is to apply Grover's Algorithm for $\mathcal{O}(\sqrt{N})$ iterations. Each iteration consists of the oracle query, which marks the optimal solution by applying a phase shift, and the Grover diffusion operator, which amplifies the amplitude of the marked state and reduces the amplitude of the non-marked states. After $\mathcal{O}(\sqrt{N})$ iterations, the probability of measuring the optimal solution becomes close to 1.

\subsection{Measuring the Quantum Register}

The final step of the algorithm is to measure the quantum register, which collapses the qubits' state to a classical bit value corresponding to the optimal solution. The measurement outcome can be decoded to obtain the optimal decision variable vector $x^*$ and the optimal objective function value $f(x^*)$.

\section{Complexity Analysis}
\label{sec:

\section{Quadratic Programming Problem Representation}

In the given problem, we consider a simple quadratic programming problem with two variables stored in registers R0 and R1. The values in R0 and R1 represent the coefficients of the quadratic and linear terms of the quadratic equation, respectively. The quadratic equation can be represented as:

\begin{equation}
f(x) = ax^2 + bx + c
\end{equation}

where $a$ and $b$ are coefficients stored in R0 and R1, respectively, and $x$ is the variable. In this specific scenario, the largest number allowed is 3, and the constant term $c$ is assumed to be 1 for simplicity.

\section{Algorithm and ARM Assembly Implementation}

The objective of the algorithm is to determine if the values stored in R0 and R1 represent a valid solution to the quadratic programming problem. To achieve this, we need to calculate the discriminant of the quadratic equation, which is given by:

\begin{equation}
\Delta = b^2 - 4ac
\end{equation}

The discriminant, $\Delta$, tells us the nature of the roots of the quadratic equation. If $\Delta \geq 0$, the equation has real roots (either one or two depending on the value of $\Delta$), and it is considered a valid solution. If $\Delta < 0$, the equation has complex roots, and it is considered an invalid solution.

To implement this algorithm in ARM assembly, we follow the given instructions and constraints. The algorithm can be broken down into three main steps:

\begin{enumerate}
\item Calculate $b^2$ and store it in a register (R2).
\item Calculate $4ac$ and store it in another register (R3).
\item Calculate the discriminant $\Delta$ by subtracting the value in R3 from the value in R2, and store the result in a new register (R4).
\end{enumerate}

For each of these steps, we use a combination of ARM assembly instructions such as MOV, LSL, and SUB to efficiently perform the required calculations. The MOV instruction is used to copy the values from R0 and R1 to R2 and R3. The LSL (logical shift left) instruction is used to multiply the values in R2 and R3 by the required factors (2 for $b^2$ and 4 for $4ac$). Finally, the SUB instruction is used to calculate the discriminant $\Delta$ by subtracting the value in R3 from the value in R2.

Once the discriminant is calculated, we need to check if it is greater than or equal to zero. To do this, we use the CMP (compare) instruction to compare the value in R4 (discriminant) with 0. Based on the result of this comparison, the ZERO PSR (Program Status Register) flag is set accordingly. If the discriminant is greater than or equal to zero, the ZERO flag will be set, indicating a valid solution.

\section{Analysis of the Algorithm}

The proposed algorithm is efficient and adheres to the given constraints. It does not use any loops or branches, ensuring a linear execution flow. Moreover, it does not use any forbidden instructions, such as MUL, MLA, or any branch instructions. Additionally, each register is used only once, and a register is not used twice in an instruction, satisfying the given register usage constraints.

The algorithm calculates the discriminant of the quadratic equation and sets the ZERO PSR flag accordingly, which can be used to determine if the solution is valid or not. This approach is suitable for simple quadratic programming problems with limited input values, as in the given scenario.

In conclusion, the algorithm effectively solves the given problem using ARM assembly instructions while adhering to the specified constraints. It provides an efficient method for determining the validity of the solution to a simple quadratic programming problem.

\section{Conclusion}
\label{sec:conclusion}

In this paper, we have proposed a novel quantum algorithm for solving the Quadratic Programming problem using Grover's Algorithm. Our method leverages the power of quantum computing to search for the optimal solution more efficiently than classical algorithms, providing a promising alternative for addressing large-scale and computationally challenging QP problems. We have provided a detailed description of the developed algorithm, analyzed its complexity, and discussed potential applications in various fields such as finance, engineering, and machine learning.

Future research directions include extending the proposed algorithm to handle non-convex QP problems and exploring other quantum optimization techniques for solving QP problems. Additionally, further investigation into the practical implementation of the proposed algorithm on quantum hardware and its integration with classical QP solvers would be valuable for realizing the full potential of quantum computing in solving optimization problems.

