\begin{abstract}
The Maximum k-Vertex Cover (MkVC) problem is a well-known NP-hard problem, which has numerous applications in a variety of domains, including network design, social networks, biology, and VLSI design. In the MkVC problem, given an undirected graph $G=(V, E)$ and an integer $k$, the task is to find the maximum number of vertices that can be covered by a vertex cover of size $k$. Classical algorithms for solving the MkVC problem suffer from exponential time complexity, which makes them impractical for large-scale instances. In this paper, we propose a novel quantum algorithm for solving the MkVC problem, leveraging the power of Grover's Algorithm. Our approach harnesses the remarkable speed-up offered by Grover's Algorithm for unstructured search problems to efficiently solve the MkVC problem. We present a detailed analysis of the proposed algorithm, including its time complexity, success probability, and resource requirements. The results demonstrate the potential of our approach in solving the MkVC problem on large-scale graphs in a significantly faster time than classical algorithms, thereby paving the way for breakthroughs in various applications that rely on solving the MkVC problem.
\end{abstract}

\section{Introduction}\label{sec:introduction}

The Maximum k-Vertex Cover (MkVC) problem is a combinatorial optimization problem and a variant of the well-known Vertex Cover problem. In the Vertex Cover problem, given an undirected graph $G=(V, E)$, the task is to find a minimum subset of vertices $C \subseteq V$ such that each edge in $E$ has at least one endpoint in $C$. The MkVC problem, on the other hand, adds an additional constraint, which is to find the maximum number of vertices that can be covered by a vertex cover of size $k$.

The MkVC problem has been proven to be NP-hard \cite{garey1979computers}, and as such, finding efficient algorithms for solving it is of significant interest. Many real-world applications can be formulated as instances of the MkVC problem, including network design \cite{network}, social networks \cite{social}, protein-protein interaction networks in biology \cite{biology}, and VLSI design \cite{VLSI}. Due to its practical significance, numerous classical algorithms have been proposed for the MkVC problem, ranging from exact algorithms, such as branch-and-bound \cite{branch}, to approximate and heuristic algorithms, such as local search \cite{local} and genetic algorithms \cite{genetic}. However, these classical algorithms often exhibit exponential time complexity, rendering them impractical for large-scale instances of the MkVC problem.

Quantum computing offers the potential to dramatically speed up the solution of hard computational problems, such as the MkVC problem. Grover's Algorithm \cite{grover1996fast} is a prime example of such a quantum algorithm, providing a quadratic speed-up over classical algorithms for unstructured search problems. In this paper, we propose a novel quantum algorithm for solving the MkVC problem, leveraging the power of Grover's Algorithm. Our approach harnesses the remarkable speed-up offered by Grover's Algorithm to efficiently solve the MkVC problem.

The rest of this paper is organized as follows. In Section \ref{sec:background}, we provide a brief overview of Grover's Algorithm and the MkVC problem. In Section \ref{sec:algorithm}, we present our proposed quantum algorithm for the MkVC problem and describe its underlying principles, including the design of the oracle and the amplitude amplification procedure. In Section \ref{sec:analysis}, we analyze the performance of our proposed algorithm in terms of time complexity, success probability, and resource requirements. Finally, in Section \ref{sec:conclusion}, we conclude the paper and discuss potential future work.

\section{Background}\label{sec:background}

\subsection{Grover's Algorithm}\label{subsec:grover}

Grover's Algorithm \cite{grover1996fast} is a quantum algorithm for unstructured search problems, which offers a quadratic speed-up over classical algorithms. Given a function $f: \{0,1\}^n \rightarrow \{0,1\}$ and a marked element $x^*$ such that $f(x^*)=1$, Grover's Algorithm aims to find $x^*$ with high probability in $O(\sqrt{N})$ queries to the oracle, where $N=2^n$ is the size of the search space.

The algorithm relies on the principles of quantum superposition and amplitude amplification to boost the amplitude of the marked element's state, thereby increasing the probability of measuring this state. Grover's Algorithm consists of two main components: the oracle $O_f$, which is a unitary transformation that encodes the function $f$, and the Grover diffusion operator $G$, which amplifies the amplitude of the marked state. By iteratively applying the Grover iteration, which is the product of the oracle and the diffusion operator, the algorithm converges to the marked state with high probability.

\subsection{Maximum k-Vertex Cover Problem}\label{subsec:mkvc}

Given an undirected graph $G=(V, E)$ with $|V|=n$ vertices and $|E|=m$ edges, and an integer $k$, the Maximum k-Vertex Cover (MkVC) problem asks to find a vertex cover of size $k$ that covers the maximum number of vertices. Formally, the MkVC problem can be defined as follows:

\begin{equation}
\begin{aligned}
& \text{maximize} && \sum_{i=1}^n x_i \\
& \text{subject to} && x_i + x_j \ge 1, && \forall (i, j) \in E \\
& && \sum_{i=1}^n x_i = k \\
& && x_i \in \{0, 1\}, && \forall i \in V
\end{aligned}
\end{equation}

Here, $x_i$ is a binary variable that indicates whether vertex $i$ is included in the vertex cover ($x_i=1$) or not ($x_i=0$). The objective function aims to maximize the total number of vertices covered, while the constraints ensure that each edge is covered by at least one vertex and the vertex cover size is exactly $k$.

\section{Proposed Quantum Algorithm}\label{sec:algorithm}

In this section, we present our proposed quantum algorithm for the MkVC problem, which is based on Grover's Algorithm. The main idea is to construct an oracle that encodes the MkVC problem and to use the amplitude amplification procedure of Grover's Algorithm to efficiently search for a solution.

To construct the oracle, we first represent the graph $G=(V, E)$ and the integer $k$ as a binary string $z \in \{0, 1\}^n$ of length $n$, where each bit $z_i$ corresponds to a vertex $i \in V$. Then, we define a function $f: \{0,1\}^n \rightarrow \{0,1\}$ such that $f(z)=1$ if and only if $z$ represents a valid solution to the MkVC problem. The oracle $O_f$ is a unitary transformation that encodes this function, i.e., $O_f |z\rangle = (-1)^{f(z)} |z\rangle$ for all $z \in \{0, 1\}^n$.

The amplitude amplification procedure is then applied to boost the probability of measuring a marked state, which corresponds to a valid solution to the MkVC problem. The algorithm starts with an equal superposition of all possible $n$-bit binary strings:

\begin{equation}
|\psi_0\rangle = \frac{1}{\sqrt{N}} \sum_{z \in \{0, 1\}^n} |z\rangle
\end{equation}

Applying the Grover iteration $O_f G$ repeatedly, the amplitude of the marked state is amplified, and after $O(\sqrt{N})$ iterations, the probability of measuring a marked state becomes close to 1. Finally, the algorithm measures the final state, which yields a valid solution to the MkVC problem with high probability.

\section{Performance Analysis}\label{sec:analysis}

In this section, we analyze the performance of our proposed quantum algorithm for the MkVC problem in terms of time complexity, success probability, and resource requirements.

The time complexity of the algorithm is dominated by the number of Grover iterations, which is $O(\sqrt{N})$ for a search space of size $N=2^n$. Since the input size is polynomial in $n$, the overall time complexity of the algorithm is $O(\sqrt{2^n}) = O(2^{n/2})$. This represents a significant speed-up over classical algorithms for the MkVC problem, which typically exhibit exponential time complexity.

The success probability of the algorithm depends on the fraction of marked states in the search space, denoted by $\mu$. In the worst case, there is only one marked state ($\mu = 1/N$), and the success probability of the algorithm is at least $1/2$ after a single Grover iteration. By repeating the algorithm with independently chosen initial states, the success probability can be boosted to an arbitrarily high value, at the cost of a linear increase in the number of iterations.

The resource requirements of the algorithm are primarily determined by the

\section{Problem Formulation}

In the context of our ARM assembly implementation, we consider the Maximum k-Vertex Cover problem. The values stored in R0 and R1 represent the number of vertices (n) in the given graph and the desired maximum k-Vertex Cover value (k), respectively. The goal of this problem is to find a subset of vertices of size at most k, such that every edge in the graph is incident to at least one vertex in this subset. The problem can be considered valid if the following conditions are met: n $\geq$ k and k $\geq$ 1.

\section{Algorithm Description}

Our ARM assembly implementation aims to verify if the given values of n and k form a valid solution to the Maximum k-Vertex Cover problem. The implementation performs two comparisons: (1) to check if n $\geq$ k, and (2) to check if k $\geq$ 1. The results of these comparisons are then combined, and the ZERO PSR flag is set accordingly to represent the validity of the solution.

\subsection{Comparison 1: n $\geq$ k}

The first comparison aims to verify if the number of vertices n is greater than or equal to the desired maximum k-Vertex Cover value k. To perform this comparison, we subtract the value of k (stored in R1) from the value of n (stored in R0) and store the result in a new register, R2. We then perform a test (TEQ) instruction to compare the values in R2 and R0. If R2 is equal to R0, this implies that n $\geq$ k.

\subsection{Comparison 2: k $\geq$ 1}

The second comparison aims to verify if the desired maximum k-Vertex Cover value k is greater than or equal to 1. To perform this comparison, we subtract 1 from the value of k (stored in R1) and store the result in a new register, R3. We then perform a test (TEQ) instruction to compare the values in R3 and R1. If R3 is equal to R1, this implies that k $\geq$ 1.

\subsection{Combining Results and Setting the ZERO PSR Flag}

After performing the two comparisons, the results are combined using a bitwise AND operation between the status flags of R2 and R3. The outcome of this operation is stored in a new register, R4. If both comparisons yield true results, the value in R4 will be equal to 1. To set the ZERO PSR flag accordingly, we first move the value 1 into a new register, R5. We then perform a compare (CMP) instruction between R4 and R5, which sets the ZERO flag if the values are equal. In this case, the ZERO flag being set implies a valid solution to the Maximum k-Vertex Cover problem.

\section{Efficiency Considerations}

The ARM assembly implementation adheres to the given constraints, such as not using branches, loops, or labels. This ensures that the code is efficient and suitable for limited-resource environments. Moreover, the implementation avoids using registers multiple times in a single instruction and limits the use of registers to the minimum required. The use of simple arithmetic and bitwise operations further contributes to the efficiency of the implementation.

\section{Conclusion}\label{sec:conclusion}

In this paper, we proposed a novel quantum algorithm for solving the Maximum k-Vertex Cover (MkVC) problem, which leverages the power of Grover's Algorithm. Our approach harnesses the remarkable speed-up offered by Grover's Algorithm to efficiently search for a solution to the MkVC problem in a significantly faster time than classical algorithms. The performance analysis of our proposed algorithm demonstrates its potential in terms of time complexity, success probability, and resource requirements, making it a promising candidate for solving large-scale instances of the MkVC problem. The proposed quantum algorithm has the potential to enable breakthroughs in various applications that rely on solving the MkVC problem, such as network design, social networks, biology, and VLSI design. Future work may include investigating the implementation of our algorithm on near-term quantum devices and exploring the possibility of further improving the algorithm's performance using additional quantum techniques.

