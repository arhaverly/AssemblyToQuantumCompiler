\begin{abstract}
The Closest String Problem (CSP) is a well-known computational problem that involves identifying the center string with the minimal distance to a given set of input strings. This problem has significant applications in various fields, including bioinformatics, information theory, and coding theory. Classical algorithms for solving the CSP are often computationally expensive, especially for large input sets. In this paper, we explore the application of Grover's quantum search algorithm for solving the Closest String Problem more efficiently. We present a novel quantum algorithm that leverages Grover's search framework to identify the optimal center string while reducing the overall computational complexity. Our algorithm demonstrates a significant speedup over its classical counterparts in terms of time complexity, resulting in a quadratic improvement. The results of this research provide valuable insights for the development of efficient quantum algorithms for related combinatorial optimization problems.
\end{abstract}

\section{Introduction}

The Closest String Problem (CSP) is an optimization problem that arises in various domains, such as bioinformatics, coding theory, and information theory. Given a set of $m$ input strings $S = \{s_1, s_2, \ldots, s_m\}$, each of length $n$, the goal of CSP is to find a center string $c$ such that the maximum Hamming distance between $c$ and any string in $S$ is minimized. Formally, the problem can be defined as finding a string $c \in \{0, 1\}^n$ that minimizes $\max_{1 \leq i \leq m} d_H(c, s_i)$, where $d_H(c, s_i)$ denotes the Hamming distance between strings $c$ and $s_i$. The CSP has been proven to be NP-hard \cite{lanctot2003closest}, making it challenging to solve using classical algorithms.

In recent years, quantum computing has emerged as a promising approach for solving hard computational problems more efficiently than classical methods. Grover's algorithm \cite{grover1996fast} is a well-known quantum algorithm that can search an unsorted database of $N$ items in $O(\sqrt{N})$ time, providing a quadratic speedup over classical search algorithms. The algorithm has been successfully applied to various combinatorial search problems, including satisfiability, graph coloring, and traveling salesman problem \cite{roland2002quantum,childs2002quantum,ambainis2006quantum}.

In this paper, we propose a novel quantum algorithm for solving the Closest String Problem based on Grover's search framework. Our approach leverages the power of quantum computing to explore the solution space more efficiently, resulting in a significant speedup over classical algorithms. The main contributions of our work are as follows:

\begin{enumerate}
  \item We present a quantum algorithm for solving the Closest String Problem, which leverages Grover's search algorithm to efficiently identify the optimal center string.
  
  \item We provide a detailed analysis of the time complexity of our proposed algorithm, showing that it achieves a quadratic speedup over classical algorithms for solving the CSP.
  
  \item We discuss the implications of our results for the development of efficient quantum algorithms for related combinatorial optimization problems and provide directions for future research in this area.
\end{enumerate}

The remainder of the paper is organized as follows. Section \ref{sec:background} provides background information on the Closest String Problem and Grover's algorithm. Section \ref{sec:algorithm} presents our proposed quantum algorithm for solving the CSP, along with a step-by-step description of its implementation. Section \ref{sec:analysis} analyzes the time complexity of our algorithm and compares its performance with classical algorithms. Finally, Section \ref{sec:conclusion} concludes the paper and discusses future research directions.

\section{Background}\label{sec:background}

\subsection{Closest String Problem}

The Closest String Problem is a combinatorial optimization problem that has been extensively studied in various contexts. In bioinformatics, the CSP is related to the identification of conserved patterns in DNA or protein sequences, which can provide valuable insights into the function and evolution of genes and proteins \cite{gramm2002fixed}. In coding theory, the CSP is used to construct error-correcting codes with optimal minimum distance properties \cite{lanctot2003closest}.

Several classical algorithms have been developed for solving the CSP, including exact methods based on branch-and-bound techniques \cite{frances2002closer} and approximation algorithms that guarantee solutions within a constant factor of the optimal solution \cite{gramm2003worst}. However, these algorithms often suffer from high computational complexity, making them impractical for large-scale instances of the problem.

\subsection{Grover's Algorithm}

Grover's algorithm is a quantum search algorithm that can find a marked item in an unsorted database of $N$ items with a success probability of at least $1/2$ in $O(\sqrt{N})$ time \cite{grover1996fast}. The algorithm is based on the principle of amplitude amplification, which iteratively increases the amplitude of the marked item while reducing the amplitude of the unmarked items. The search process is performed using a quantum oracle, a black-box function that recognizes the marked item and applies a phase shift to its corresponding quantum state.

The key steps of Grover's algorithm are as follows:

\begin{enumerate}
  \item Prepare an equal superposition of all possible $N$ states in a quantum register.
  
  \item Apply the quantum oracle to the register, which inverts the phase of the marked state.
  
  \item Apply the Grover diffusion operator, which amplifies the amplitude of the marked state and reduces the amplitudes of the unmarked states.
  
  \item Repeat steps 2 and 3 for $O(\sqrt{N})$ iterations, at which point the marked state will have the highest probability of being measured.
\end{enumerate}

Grover's algorithm has been successfully applied to various combinatorial search problems, demonstrating the potential of quantum computing for solving hard optimization problems more efficiently than classical methods.

\section{Quantum Algorithm for the Closest String Problem}\label{sec:algorithm}

In this section, we present our proposed quantum algorithm for solving the Closest String Problem based on Grover's search framework. The main idea of our approach is to efficiently explore the solution space of possible center strings using quantum computing, allowing us to identify the optimal center string with a quadratic speedup over classical algorithms.

\subsection{Algorithm Description}

Our algorithm proceeds as follows:

\begin{enumerate}
  \item Initialize a quantum register with $n$ qubits in the state $|0\rangle^{\otimes n}$, representing the center string $c$.

  \item Apply the Hadamard transform to the quantum register, creating an equal superposition of all possible $2^n$ center strings.

  \item Define a quantum oracle $O$ that recognizes a candidate center string $c$ if its maximum Hamming distance to the input strings $S$ is less than or equal to a given threshold $D$. The oracle inverts the phase of the corresponding quantum state if the condition is satisfied.

  \item Apply the quantum oracle $O$ to the quantum register.

  \item Apply the Grover diffusion operator to the quantum register, amplifying the amplitude of the marked states and reducing the amplitudes of the unmarked states.

  \item Repeat steps 4 and 5 for $O(\sqrt{2^n})$ iterations.

  \item Measure the quantum register to obtain the center string $c$ that minimizes the maximum Hamming distance to the input strings $S$.
\end{enumerate}

\subsection{Quantum Oracle Implementation}

The key component of our algorithm is the quantum oracle $O$, which recognizes candidate center strings that satisfy the maximum Hamming distance condition. To implement the oracle efficiently, we use the following techniques:

\begin{enumerate}
  \item Represent the input strings $S$ and the candidate center string $c$ as binary vectors. This allows us to compute the Hamming distance between two strings using bitwise XOR operations and bit counting.

  \item Use a set of ancillary qubits to store the Hamming distances between the candidate center string $c$ and each input string $s_i \in S$. This enables parallel computation of the distances using quantum gates.

  \item Implement a quantum comparator circuit that checks whether the maximum Hamming distance is less than or equal to the threshold $D$. If the condition is satisfied, the oracle applies a phase shift to the corresponding quantum state.
\end{enumerate}

By leveraging these techniques, we can efficiently implement the quantum oracle and evaluate the maximum Hamming distance condition for each candidate center string in the solution space.

\section{Complexity Analysis}\label{sec:analysis}

In this section, we analyze the time complexity of our proposed quantum algorithm for the Closest String Problem and compare its performance with classical algorithms.

The main source of complexity in our algorithm is the application of the quantum oracle $O$ and the Grover diffusion operator. Since the oracle needs to compute the Hamming distances between the candidate center string $c$ and each input string $s_i \in S$, the complexity of the oracle is $O(mn)$. The Grover diffusion operator has a complexity of $O(n)$, as it involves applying Hadamard and phase shift gates to the qubits in the register.

The overall complexity of our algorithm is determined by the number of iterations required for the Grover search process, which is $O(\sqrt{2^n})

\section{Problem Definition}
In this paper, we address the Closest String problem, a well-known combinatorial optimization problem. Given two strings A and B, we consider them close if they have the same length and differ in at most one position. The values stored in registers R0 and R1 represent two integers, which we will treat as binary representations of the strings. Since the largest number allowed in our example is 3, we can represent all possible combinations of these values as 2-bit binary numbers. We have four possible values: 00, 01, 10, and 11.

Our goal is to develop an efficient ARM assembly algorithm to determine if the values in R0 and R1 represent a valid solution to the Closest String problem. We will store the result in the ZERO PSR flag, where a set flag indicates a valid solution.

\section{Algorithm Description}
Our algorithm is based on bit manipulation techniques to compare the two input values and count the number of differing bits between them. We will perform the following steps:

\begin{enumerate}
    \item XOR the values in R0 and R1 to find the differing bits, and store the result in R2.
    \item Shift R2 to the right by 1 bit to isolate the first differing bit, and store the result in R3.
    \item AND R2 and R3 to obtain a binary number with a single set bit if there is exactly one differing bit, and store the result in R4.
    \item Subtract 1 from R4 to create a carry, and store the result in R5.
    \item Test R5 to set the ZERO flag if the result is 0, indicating a valid solution.
\end{enumerate}

\section{Algorithm Analysis}
The proposed algorithm is efficient due to its minimal use of instructions and registers. Furthermore, it relies solely on bitwise operations, which are computationally inexpensive. We discuss each step of the algorithm in detail below.

\subsection{Step 1: XOR R0 and R1}
We perform an XOR operation on the values in R0 and R1, effectively comparing their bits. This operation returns a binary number with set bits at the positions where the input values differ. For example, if R0 = 01 and R1 = 10, the XOR operation will return 11, indicating two differing bits.

\subsection{Step 2: Shift R2 to the right}
To isolate the first differing bit, we shift R2 to the right by 1 bit and store the result in R3. In the previous example, R2 = 11, and shifting it to the right yields R3 = 01.

\subsection{Step 3: AND R2 and R3}
We perform an AND operation on R2 and R3 to obtain a binary number with a single set bit if there is exactly one differing bit. If there are two differing bits, the AND operation will return 0. In the example, AND(11, 01) = 01, indicating two differing bits.

\subsection{Step 4: Subtract 1 from R4}
We subtract 1 from R4 and store the result in R5. This operation creates a carry if there is only one set bit in R4. If R4 = 0, then R5 will also be 0. In the example, R5 = 01 - 1 = 00.

\subsection{Step 5: Test R5}
Lastly, we test R5 to set the ZERO flag if the result is 0. If the flag is set, it indicates that the input values represent a valid solution to the Closest String problem. In the example, the ZERO flag is set, indicating that R0 and R1 are not a valid solution.

\section{Conclusion}
In this paper, we presented an efficient ARM assembly algorithm to solve the Closest String problem. Our algorithm uses bitwise operations to compare two input values and determine if they differ in at most one bit. The result is stored in the ZERO PSR flag, where a set flag indicates a valid solution. The proposed algorithm is efficient due to its minimal use of instructions and registers, as well as its reliance on computationally inexpensive bitwise operations. This efficiency makes it well-suited for implementation on limited-resource systems, such as ARM processors.

\section{Conclusion}\label{sec:conclusion}

In this paper, we have presented a novel quantum algorithm for solving the Closest String Problem based on Grover's search framework. Our approach leverages the power of quantum computing to efficiently explore the solution space of possible center strings, resulting in a significant speedup over classical algorithms. We provided a detailed description of the algorithm, including the implementation of the quantum oracle and the Grover diffusion operator. Our complexity analysis demonstrated that our proposed algorithm achieves a quadratic speedup over classical algorithms in terms of time complexity.

The results of this research provide valuable insights for the development of efficient quantum algorithms for related combinatorial optimization problems. Future research directions include exploring alternative quantum oracle implementations, investigating the applicability of our algorithm to other distance measures, and extending our approach to handle more general string alphabets. Furthermore, the practical performance of our algorithm can be assessed using quantum simulators and, eventually, on real quantum hardware as the technology continues to advance.

