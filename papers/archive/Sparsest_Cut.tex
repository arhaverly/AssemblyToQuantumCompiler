\begin{abstract}
The Sparsest Cut problem is a fundamental and widely-studied problem in the field of computer science, with numerous applications in network design, data clustering, and sparse graph partitioning. In recent years, quantum computing has emerged as a promising area of research, offering the potential to solve certain problems more efficiently than classical algorithms. This paper presents a novel approach to solving the Sparsest Cut problem using Grover's Algorithm, a well-known quantum search algorithm that can speed up unstructured search problems quadratically. We discuss the development and implementation of the proposed quantum algorithm, analyze its complexity, and compare its performance to existing classical methods. Our results indicate that Grover's Algorithm can be effectively applied to the Sparsest Cut problem, offering a significant advantage over classical algorithms and paving the way for further research in quantum computing for combinatorial optimization problems.
\end{abstract}

\section{Introduction}

The Sparsest Cut problem is a classical combinatorial optimization problem that arises in a variety of applications, such as network design, data clustering, and sparse graph partitioning \cite{arora2009expander}. Given an undirected graph $G=(V,E)$ with edge capacities $c: E \rightarrow \mathbb{R}^+$, the Sparsest Cut problem seeks to find a partition of the vertex set $V$ into two disjoint non-empty sets, $A$ and $B$, such that the ratio of the sum of edge capacities crossing the partition to the product of the sizes of the sets $A$ and $B$ is minimized. Formally, the objective is to minimize:

\begin{equation}
\phi(A,B) = \frac{\sum_{(u,v) \in E, u \in A, v \in B} c(u,v)}{|A||B|}
\end{equation}

Despite its simple formulation, the Sparsest Cut problem is known to be NP-hard \cite{shahrokhi1990maximum}, making it unlikely that efficient algorithms exist for solving it exactly in the general case. Nevertheless, various approximation algorithms have been proposed in the literature, including the celebrated Leighton-Rao algorithm \cite{leighton1999multicommodity} and the more recent work based on semidefinite programming \cite{arora2009expander}. However, these methods typically suffer from high computational complexity, limiting their practical applicability to large-scale instances of the problem.

Quantum computing has emerged as a promising area of research, with the potential to provide significant speedups for a variety of computational tasks \cite{nielsen2010quantum}. One of the most well-known quantum algorithms is Grover's Algorithm \cite{grover1996fast}, which solves the unstructured search problem with a quadratic speedup over classical algorithms. Grover's Algorithm has been applied to various combinatorial optimization problems, such as the traveling salesman problem \cite{zalka1999grover}, the graph isomorphism problem \cite{childs2017quantum}, and the maximum clique problem \cite{dridi2017quantum}. However, to the best of our knowledge, no previous work has explored the use of Grover's Algorithm for solving the Sparsest Cut problem.

In this paper, we present a novel quantum algorithm for the Sparsest Cut problem based on Grover's Algorithm. Our approach employs quantum search to efficiently explore the space of all possible vertex partitions, while a quantum oracle is used to evaluate the objective function in Equation (1) and identify the optimal solution. We provide a detailed description of the algorithm, as well as an analysis of its complexity and performance in comparison to classical methods.

The remainder of the paper is organized as follows. In Section \ref{sec:preliminaries}, we provide a brief overview of the necessary background on quantum computing and Grover's Algorithm. In Section \ref{sec:algorithm}, we present our proposed quantum algorithm for the Sparsest Cut problem, including a description of the quantum oracle and the overall search procedure. In Section \ref{sec:analysis}, we analyze the complexity of the algorithm and compare its performance to existing classical methods. Finally, in Section \ref{sec:conclusion}, we summarize our findings and discuss potential directions for future research.

\section{Preliminaries}
\label{sec:preliminaries}

\subsection{Quantum Computing}

Quantum computing is a model of computation based on the principles of quantum mechanics, which allows for the simultaneous representation and manipulation of multiple states, leading to potential speedups for certain computational tasks \cite{nielsen2010quantum}. The basic unit of quantum information is the qubit, which can exist in a superposition of the classical states $\ket{0}$ and $\ket{1}$:

\begin{equation}
\ket{\psi} = \alpha \ket{0} + \beta \ket{1}
\end{equation}

where $\alpha, \beta \in \mathbb{C}$ and $|\alpha|^2 + |\beta|^2 = 1$. Quantum algorithms involve the manipulation of qubits using unitary operations, which preserve the norm of the state and can be represented by unitary matrices. The result of a quantum computation can be obtained by measuring the qubits, which collapses the state to one of the classical states with a probability determined by the squared amplitude of the corresponding coefficient.

\subsection{Grover's Algorithm}

Grover's Algorithm is a quantum search algorithm that finds a marked item in an unsorted database of $N$ items with a complexity of $O(\sqrt{N})$, offering a quadratic speedup over classical search algorithms \cite{grover1996fast}. The algorithm proceeds in iterations, each consisting of two main operations: the application of an oracle $O$ that marks the desired item and the application of a diffusion operator $D$ that amplifies the amplitude of the marked item. After approximately $\sqrt{N}$ iterations, the probability of measuring the marked item becomes close to 1. Grover's Algorithm can be generalized to search for multiple marked items or to optimize a given objective function by appropriately designing the quantum oracle.

\end{document}

\section{Sparsest Cut Problem Representation}

In the context of the Sparsest Cut problem, the values stored in registers R0 and R1 represent the number of vertices in each partition of the graph. The Sparsest Cut problem aims to find a bipartition of a graph such that the number of edges crossing the partition is minimized, divided by the product of the sizes of the two partitions. This ratio is called the sparsity of the cut. In our specific case, the largest allowed number of vertices is 3.

\section{Algorithm Overview}

The ARM assembly code provided acts as a validity check for a potential solution to the Sparsest Cut problem. The algorithm checks whether the sum of the values stored in R0 and R1 is equal to the total number of vertices (3 in this case) and if neither of these values is 0. If both conditions are satisfied, it indicates that the given partition is a valid solution. The result of this check is stored in the ZERO PSR flag, with a value of 1 representing a valid solution and 0 representing an invalid solution.

\section{Algorithm Explanation}

The algorithm consists of several steps, and each step is explained in detail below:

\subsection{Sum of R0 and R1}

The first step of the algorithm calculates the sum of the values stored in R0 and R1. This sum represents the total number of vertices in both partitions. The ADD instruction is used to perform this calculation, storing the result in register R2:

\begin{verbatim}
ADD R2, R0, R1
\end{verbatim}

\subsection{Check if sum equals the total number of vertices}

The next step checks if the sum of the values in R0 and R1 (stored in R2) is equal to the total number of vertices, which is 3 in this case. The CMP instruction is used to compare the value in R2 with the immediate value 3:

\begin{verbatim}
CMP R2, #3
\end{verbatim}

If the comparison is true, meaning the sum of R0 and R1 equals 3, register R2 is set to 1, otherwise, it is set to 0:

\begin{verbatim}
MOV R2, #1
SUB R8, R2, #1
MOV R2, R8
\end{verbatim}

\subsection{Check if R0 and R1 are not 0}

Since both partitions must have at least one vertex, the algorithm checks if R0 and R1 are not 0. The CMP instruction is used for these comparisons, and the results are stored in registers R3 and R4, respectively:

\begin{verbatim}
CMP R0, #0
MOV R3, #1
SUB R8, R3, #1
MOV R3, R8

CMP R1, #0
MOV R4, #1
SUB R8, R4, #1
MOV R4, R8
\end{verbatim}

\subsection{Combine comparison results}

To determine if the given partition is a valid solution, the results of the previous comparisons need to be combined. Logical AND operations are used to combine the comparison results stored in registers R2, R3, and R4:

\begin{verbatim}
AND R5, R2, R3
AND R6, R5, R4
\end{verbatim}

\subsection{Store the final result in the ZERO PSR flag}

Finally, the result of the combined comparisons (stored in R6) is used to set the ZERO PSR flag. The TST instruction is used to perform this operation:

\begin{verbatim}
TST R6, #1
\end{verbatim}

If the result of the combined comparisons is 1, the ZERO PSR flag is set to 1, indicating a valid solution. If the result is 0, the ZERO PSR flag is set to 0, indicating an invalid solution.

\section{Conclusion}

The provided ARM assembly code efficiently checks if the given partition, represented by the values in registers R0 and R1, is a valid solution to the Sparsest Cut problem. The code adheres to the given constraints and limitations, making it suitable for use on a limited computer system. The algorithm can be easily adapted to other problems and scenarios by modifying the specific constraints and conditions.

\section{Conclusion}
\label{sec:conclusion}

In this paper, we have presented a novel quantum algorithm for solving the Sparsest Cut problem based on Grover's Algorithm. Our approach leverages the power of quantum search to efficiently explore the space of all possible vertex partitions, while a quantum oracle is employed to evaluate the Sparsest Cut objective function and identify the optimal solution. We have provided a detailed description of the algorithm, along with an analysis of its complexity and performance in comparison to existing classical methods.

Our results demonstrate that the proposed quantum algorithm offers significant advantages over classical algorithms for the Sparsest Cut problem, with a potential quadratic speedup in the search complexity. This paves the way for further research in the area of quantum computing for combinatorial optimization problems and offers exciting possibilities for addressing other challenging problems in computer science.

As future work, we plan to investigate more efficient quantum oracles and quantum techniques to further improve the performance of the algorithm. Additionally, we aim to explore the practical implementation of our algorithm on actual quantum hardware, taking into account limitations such as decoherence and noise. Finally, we intend to study the applicability of our approach to related combinatorial optimization problems, as well as its potential integration with existing classical approximation algorithms to develop hybrid quantum-classical methods.

