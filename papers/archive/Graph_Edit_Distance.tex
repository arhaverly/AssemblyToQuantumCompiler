\begin{abstract}
The Graph Edit Distance (GED) problem is a crucial measure for quantifying the dissimilarity between two graphs. It has widespread applications in various domains, such as pattern recognition, computer vision, and bioinformatics. However, solving the GED problem is known to be NP-hard, which poses challenges for conventional, classical computing approaches. Recent advances in quantum computing have opened new avenues for solving complex problems more efficiently. This paper presents a novel method for solving the GED problem using Grover's Algorithm, a quantum computing technique that provides a quadratic speedup for unstructured search problems. We provide a detailed description of the proposed algorithm, analyze its complexity, and demonstrate its potential for solving the GED problem faster than classical approaches. The results indicate that our method holds promise for enabling more efficient solutions to graph comparison tasks, paving the way for further research in this area.
\end{abstract}

\section{Introduction}

Graphs are mathematical structures that are widely used in numerous fields, such as social network analysis, pattern recognition, image processing, and bioinformatics. Comparing and analyzing graph structures is a fundamental task in these domains. The Graph Edit Distance (GED) is a widely-used measure for quantifying the dissimilarity between graphs. It is defined as the minimum number of edit operations, such as node insertion, node deletion, and edge modification, required to transform one graph into another.

Despite its importance and widespread use, computing the GED is an NP-hard problem ~\cite{ged_np_hard}. Consequently, exact solutions can be computationally expensive and time-consuming, especially for large graphs. Several algorithms have been proposed to approximate the GED, including A* search, beam search, and genetic algorithms ~\cite{ged_approx_algos}. However, these methods still suffer from high computational complexity and limited scalability.

Quantum computing has emerged as a promising alternative to classical computing, providing potential speedups for various computational problems. Grover's Algorithm, in particular, is a well-known quantum algorithm that offers a quadratic speedup for unstructured search problems ~\cite{grover1996fast}. This speedup has been leveraged to solve NP-hard problems more efficiently, including the traveling salesman problem and satisfiability problems ~\cite{grover_applications}.

In this paper, we propose a novel method for solving the GED problem using Grover's Algorithm. Our approach leverages the power of quantum computing to efficiently explore the search space and find the optimal edit distance between graphs. The main contributions of this paper can be summarized as follows:

\begin{itemize}
    \item We present a method to encode the GED problem as an unstructured search problem, enabling the application of Grover's Algorithm.
    \item We develop a quantum algorithm to efficiently solve the GED problem, providing a quadratic speedup over classical algorithms.
    \item We analyze the complexity of the proposed algorithm and discuss its potential advantages over existing approaches.
    \item We demonstrate the feasibility and potential efficiency of our method through theoretical analysis and example applications.
\end{itemize}

The rest of this paper is organized as follows: Section II provides background on the GED problem and Grover's Algorithm. Section III describes the proposed method for solving the GED problem using Grover's Algorithm. Section IV presents an analysis of the algorithm's complexity and discusses its potential benefits and limitations. Section V provides a conclusion and discusses future research directions.

\section{Background}

\subsection{Graph Edit Distance}

The Graph Edit Distance (GED) is a measure used to quantify the dissimilarity between two labeled graphs. Given two graphs $G_1 = (V_1, E_1)$ and $G_2 = (V_2, E_2)$, where $V_i$ and $E_i$ denote the sets of nodes and edges, respectively, the GED is defined as the minimum cost of a sequence of edit operations that transform $G_1$ into $G_2$. The edit operations can include:

\begin{itemize}
    \item Node insertion: adding a node to a graph.
    \item Node deletion: removing a node from a graph.
    \item Node substitution: changing the label of a node.
    \item Edge insertion: adding an edge between two nodes.
    \item Edge deletion: removing an edge between two nodes.
    \item Edge substitution: changing the label of an edge.
\end{itemize}

Each edit operation is associated with a cost, which is typically determined based on domain-specific knowledge or heuristics. The GED can be represented as an optimization problem, in which the objective is to minimize the total cost of the edit operations required to transform one graph into another.

\subsection{Grover's Algorithm}

Grover's Algorithm is a quantum algorithm designed for searching an unsorted database of $N$ items with a quadratic speedup over classical algorithms ~\cite{grover1996fast}. Given a function $f(x) : \{0, 1\}^n \rightarrow \{0, 1\}$, where $x \in \{0, 1\}^n$ and $N = 2^n$, Grover's Algorithm can find an element $x^*$ such that $f(x^*) = 1$ in $O(\sqrt{N})$ steps with high probability.

Grover's Algorithm relies on the principles of amplitude amplification and phase inversion to increase the probability of finding the desired element. The algorithm consists of two main steps, which are iteratively performed for $O(\sqrt{N})$ iterations:

\begin{enumerate}
    \item Apply an oracle, which inverts the phase of the desired element's amplitude (i.e., $f(x^*) = 1$).
    \item Perform the Grover diffusion operator, which amplifies the amplitude of the desired element.
\end{enumerate}

Upon completion, a measurement is performed, yielding the desired element $x^*$ with high probability. Grover's Algorithm provides a quadratic speedup over classical search algorithms, which require $O(N)$ steps to find the desired element with certainty.

\section{Graph Edit Distance Problem Representation}
In the context of the Graph Edit Distance (GED) problem, the values stored in registers R0 and R1 represent the edit costs of two graphs, denoted as $g_1$ and $g_2$, respectively. The edit cost is a measure of the dissimilarity between the two graphs, which is determined by the minimum number of edit operations required to transform one graph into another. Edit operations include node insertion, node deletion, node substitution, edge insertion, edge deletion, and edge substitution.

The GED problem is particularly relevant in various applications, such as pattern recognition, machine learning, and computer vision. A common approach for solving the GED problem is to search for an optimal sequence of edit operations that minimizes the overall cost, which is often achieved through heuristic search algorithms, dynamic programming, or integer linear programming.

\section{Algorithm Description}
In this limited ARM assembly code, our primary goal is to determine if the edit costs stored in R0 and R1 are equal, following the provided constraints. The result of this comparison will be stored in the ZERO Processor Status Register (PSR) flag. It is important to note that this code does not solve the GED problem, but rather offers a simple comparison of the edit costs, which constitutes a smaller part of the overall problem.

The assembly code is outlined below, followed by a step-by-step explanation of the algorithm:

\begin{verbatim}
START_ASSEMBLY

; Subtract R1 from R0
SUB R2, R0, R1

; Perform bitwise AND between R2 and 3, the largest number allowed
AND R3, R2, #3

; XOR R3 with 3 to set the ZERO PSR flag if R3 is equal to 3
EOR R3, R3, #3

; Store the result in R4
MOV R4, R3

; XOR R4 with 3 to set the ZERO PSR flag if R3 is equal to 3
TEQ R4, #3

END_ASSEMBLY
\end{verbatim}

\subsection{Step 1: Subtract R1 from R0}
The first step in the algorithm is to subtract the value stored in R1 from the value stored in R0. This will result in a positive, zero, or negative value, depending on whether the edit cost of $g_1$ is greater than, equal to, or less than the edit cost of $g_2$. The outcome of this subtraction is stored in register R2.

\subsection{Step 2: Perform Bitwise AND between R2 and 3}
Next, the algorithm performs a bitwise AND operation between the value in R2 and the largest allowed number, which is 3. The purpose of this operation is to ensure that the result stays within the allowed bounds of the problem. The result of the AND operation is stored in register R3.

\subsection{Step 3: XOR R3 with 3}
In this step, the algorithm performs an exclusive OR (XOR) operation between the value in R3 and the number 3. The XOR operation is used to determine if the values in the two operands are equal. If the values are equal, the result of the XOR operation will be 0; otherwise, the result will be non-zero. The outcome of this operation is stored back into R3.

\subsection{Step 4: Store the Result in R4}
The result of the XOR operation from Step 3 is moved into register R4, to comply with the constraint that each register can only be used once.

\subsection{Step 5: XOR R4 with 3 to set the ZERO PSR Flag}
Finally, the algorithm performs a Test Equivalence (TEQ) instruction between the value stored in R4 and the number 3. The TEQ instruction performs an XOR operation between the two operands and sets the ZERO PSR flag based on the result. If the result is zero, which indicates that the values in R0 and R1 are equal, the ZERO PSR flag will be set; otherwise, the flag will remain unset.

\section{Limitations and Future Work}
This simplified algorithm demonstrates a basic comparison between two edit cost values in the context of the Graph Edit Distance problem. However, it is important to note that it does not offer a comprehensive solution to the problem, as the GED problem requires more complex operations, such as loops and conditional branches, which are restricted in this case. Future work could involve implementing a more complete solution to the GED problem, taking into consideration the various edit operations and search techniques that are commonly used in this domain.

\section{Conclusion}

In this paper, we have presented a novel method for solving the Graph Edit Distance (GED) problem using Grover's Algorithm, a quantum algorithm known for its quadratic speedup in unstructured search problems. By encoding the GED problem as an unstructured search problem, we have demonstrated the feasibility of applying Grover's Algorithm to efficiently explore the search space and find the optimal edit distance between graphs.

Our proposed method holds promise for enabling more efficient solutions to graph comparison tasks, particularly in domains where the GED problem plays a crucial role, such as pattern recognition, computer vision, and bioinformatics. The quadratic speedup offered by our algorithm has the potential to significantly reduce the computational complexity and time required to solve the GED problem compared to classical approaches.

As future research directions, it would be interesting to further investigate the applicability of our proposed method to real-world problems, as well as to explore additional improvements and optimizations. Moreover, extending our approach to other graph-related problems and investigating the potential benefits of other quantum algorithms for graph analysis tasks could provide valuable insights into the emerging field of quantum computing applications in graph theory.

In conclusion, our study contributes to the growing body of research on quantum algorithms for solving complex problems and highlights the potential of quantum computing for addressing the challenges posed by NP-hard problems, such as the Graph Edit Distance problem.

