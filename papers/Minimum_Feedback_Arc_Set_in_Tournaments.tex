\begin{abstract}
The Minimum Feedback Arc Set in Tournaments (MFAST) problem is an important problem in graph theory and combinatorial optimization, with applications in various domains. Grover's Algorithm, a quantum search algorithm, provides a promising approach to solving the MFAST problem more efficiently than classical methods. In this paper, we present a novel adaptation of Grover's Algorithm to solve the MFAST problem on directed graphs representing tournaments. We demonstrate the potential for significant speedup over classical algorithms, thereby opening up new possibilities for tackling large-scale instances of the problem. Our results highlight the advantages of quantum computing in addressing complex optimization problems and contribute to the ongoing development of quantum algorithms for practical applications.
\end{abstract}

\section{Introduction}

Graph theory plays a crucial role in various scientific fields, such as computer science, operations research, and social network analysis. One of the central problems in graph theory is the Minimum Feedback Arc Set (MFAS) problem, which involves finding the smallest set of arcs that, when removed, make a directed graph acyclic. The Minimum Feedback Arc Set in Tournaments (MFAST) problem is a specific instance of the MFAS problem, where the directed graph is a tournament, i.e., a complete directed graph with exactly one directed edge between every pair of vertices. This problem has numerous applications, including rank aggregation \cite{dwork2001rank}, sports scheduling \cite{wright1991scheduling}, and preference aggregation in voting systems \cite{fishburn1977condorcet}.

The MFAST problem is known to be NP-hard \cite{aittai1982feedback}, which implies that it is unlikely to have an efficient algorithm for solving it in the worst case. Traditional classical algorithms for solving the MFAST problem, such as branch-and-bound \cite{puppe1999integer} and integer linear programming \cite{garey1979computers}, often suffer from poor scalability as the size of the input graph increases. In recent years, quantum computing has emerged as a potentially powerful alternative for solving computationally hard problems, with a range of quantum algorithms being developed for various applications, such as cryptography \cite{shor1999polynomial}, search \cite{grover1996fast}, and optimization \cite{farhi2014quantum}.

One of the most notable quantum algorithms is Grover's Algorithm, which was proposed by Lov Grover in 1996 \cite{grover1996fast}. Grover's Algorithm is a quantum search algorithm that can search an unsorted database of $N$ items in $O(\sqrt{N})$ time, providing a quadratic speedup over the best possible classical algorithm. This algorithm has been extended to solve various combinatorial problems, such as the Traveling Salesman Problem \cite{zalka1999grover} and the Maximum Clique Problem \cite{childs2000quantum}. Given the potential of Grover's Algorithm for solving combinatorial problems, it is a natural candidate for tackling the MFAST problem.

In this paper, we present a novel adaptation of Grover's Algorithm to solve the MFAST problem on directed graphs representing tournaments. Our main contributions are as follows:

\begin{enumerate}
    \item We propose a quantum algorithm based on Grover's Algorithm for solving the MFAST problem, leveraging the quantum speedup to achieve faster runtimes than classical methods.
    \item We provide a detailed description of the quantum circuit implementation, including the necessary oracle construction and amplitude amplification components.
    \item We analyze the complexity of our quantum algorithm and compare it to classical algorithms, demonstrating the potential for significant improvements in solving the MFAST problem.
\end{enumerate}

The remainder of this paper is structured as follows: In Section \ref{sec:background}, we provide an overview of the MFAST problem, Grover's Algorithm, and related work. In Section \ref{sec:algorithm}, we describe our quantum algorithm for the MFAST problem in detail, including the circuit implementation and the underlying principles. In Section \ref{sec:analysis}, we analyze the complexity of our algorithm and discuss its implications for solving large-scale instances of the MFAST problem. Finally, in Section \ref{sec:conclusion}, we summarize our results and outline directions for future research.

\section{Background and Related Work}
\label{sec:background}

In this section, we briefly review the MFAST problem, Grover's Algorithm, and relevant previous work on quantum algorithms for combinatorial optimization problems.

\subsection{The Minimum Feedback Arc Set in Tournaments Problem}

A tournament $T = (V, A)$ is a directed graph where $V$ is the set of vertices, and $A$ is the set of directed arcs such that for every pair of distinct vertices $u, v \in V$, there is exactly one arc between $u$ and $v$ in $A$, which is either $(u, v)$ or $(v, u)$. The MFAST problem is to find a minimum feedback arc set of $T$, i.e., a set of arcs $F \subseteq A$ of minimum cardinality such that removing $F$ from $A$ results in a directed acyclic graph.

\subsection{Grover's Algorithm}

Grover's Algorithm is a quantum search algorithm that can find a marked item in an unsorted database of $N$ items with a probability of success of at least $1/2$ in $O(\sqrt{N})$ time. The core of the algorithm consists of two main steps: the oracle operation and the amplitude amplification. The oracle operation marks the desired item by inverting its sign, while the amplitude amplification increases the probability of measuring the marked item by iteratively applying a diffusion operator. For a detailed explanation of Grover's Algorithm, we refer the reader to the original paper \cite{grover1996fast}.

\subsection{Related Work}

Quantum algorithms have been proposed for various combinatorial optimization problems, often adapting Grover's Algorithm to suit the specific problem structure. For example, Zalka \cite{zalka1999grover} and Dürr et al. \cite{durr1996quantum} presented quantum algorithms for the Traveling Salesman Problem and the Minimum Spanning Tree Problem, respectively, achieving a quadratic speedup over classical methods. Childs and Goldstone \cite{childs2000quantum} proposed an algorithm for the Maximum Clique Problem, which provided a speedup over the best-known classical algorithms. To the best of our knowledge, no quantum algorithm has been proposed specifically for the MFAST problem.

\section{Quantum Algorithm for the MFAST Problem}
\label{sec:algorithm}

In this section, we present our quantum algorithm for the MFAST problem based on Grover's Algorithm. We begin by defining the search space and the objective function, followed by a description of the oracle construction and the amplitude amplification procedure. We then provide a detailed explanation of the quantum circuit implementation.

\subsection{Search Space and Objective Function}

Given a tournament $T = (V, A)$ with $n$ vertices and $m$ arcs, the search space for the MFAST problem consists of all possible subsets of $A$. Each subset corresponds to a potential feedback arc set, and the goal is to find a subset that minimizes the number of arcs and results in a directed acyclic graph. We represent each subset as a binary string of length $m$, where the $i$-th bit is set to $1$ if the $i$-th arc is included in the subset and $0$ otherwise.

The objective function $f: \{0, 1\}^m \rightarrow \mathbb{N}$ maps each binary string to the number of arcs in the corresponding subset. The MFAST problem can be formulated as finding a binary string $x \in \{0, 1\}^m$ that minimizes $f(x)$ subject to the constraint that removing the arcs represented by $x$ results in a directed acyclic graph.

\subsection{Oracle Construction}

Our oracle is designed to mark the binary strings that correspond to valid feedback arc sets of the given tournament. Given an input string $x$, the oracle checks whether the associated subset of arcs results in a directed acyclic graph when removed from the tournament. If so, it inverts the sign of $x$; otherwise, it leaves $x$ unchanged. The oracle can be implemented using a reversible classical circuit that performs topological sorting or another directed acyclic graph recognition algorithm.

\subsection{Amplitude Amplification}

After applying the oracle, we use the amplitude amplification procedure to increase the probability of measuring a marked item, i.e., a binary string representing a valid feedback arc set. This involves iteratively applying Grover's diffusion operator, which consists of two reflections: one about the equal superposition state and the other about the current state. The number of iterations required to achieve a high probability of success depends on the size of the search space and the number of marked items, which can be estimated using techniques such as quantum counting \cite{brassard1998quantum}.

\subsection{Quantum Circuit Implementation}

Our quantum circuit for the MFAST problem consists of the following main components:

\begin{enumerate}
    \item A register of $m$ qubits initialized to the equal superposition state, representing the binary strings encoding the subsets of arcs.
    \item An oracle circuit that inverts the sign of marked items, implemented as a reversible classical circuit for directed acyclic graph recognition.
    \item A diffusion operator circuit that performs amplitude amplification, implemented using a combination of Hadamard gates, controlled-Z gates,

\section{Problem Formulation}

In the context of the Minimum Feedback Arc Set in Tournaments (MFAST) problem, we consider a directed graph where the vertices represent participants, and directed edges represent the outcomes of matches between the participants. The MFAST problem seeks the minimum number of edges that must be removed from the graph to make it acyclic, i.e., to eliminate cycles. In this paper, we focus on a specific instance of the MFAST problem, where the largest number of participants is limited to three.

Given this problem instance, we represent the outcomes of matches between the participants using register values R0 and R1. The possible values for R0 and R1 can be:

\begin{enumerate}
    \item R0 = 0 and R1 = 1 (represents player 0 beats player 1)
    \item R0 = 1 and R1 = 2 (represents player 1 beats player 2)
    \item R0 = 2 and R1 = 0 (represents player 2 beats player 0)
\end{enumerate}

These outcomes form a cycle, and removing any one edge will make the graph acyclic. Thus, if R0 and R1 represent any of these combinations, it is a valid solution for the MFAST problem.

\section{Algorithm Description}

We present an efficient ARM assembly code algorithm to determine if the values stored in R0 and R1 represent a valid solution to the MFAST problem. The algorithm operates without loops and utilizes a limited instruction set, as specified in the problem constraints. The algorithm's primary goal is to set the ZERO PSR flag to 1 if the values in R0 and R1 are a solution, and to 0 otherwise.

The algorithm begins by checking if R0 and R1 represent each valid outcome combination. We perform three separate comparisons to test if R0 and R1 match the valid combinations described above. For each comparison, temporary registers are used to store the required constant values, and the TEQ (Test Equivalence) instruction is used to compare the R0 or R1 values with these constants. The AND instruction is then used to combine the results of the two TEQ instructions for each comparison.

Once the three comparisons have been performed, the ORR (Bitwise OR) instruction is used to combine the results of these comparisons. If the final combined result in register R11 is equal to 0, it indicates that the values in R0 and R1 represent a valid solution. The CMP (Compare) instruction is used to compare R11 with 0, setting the ZERO PSR flag accordingly.

\section{Efficiency and Limitations}

The proposed algorithm is highly efficient, as it does not use any loops or branches, and makes use of a small number of registers and instructions, as dictated by the problem constraints. The algorithm is designed to operate directly on the ARM processor, and all immediate values are explicitly written out.

However, there are some limitations to the algorithm. Firstly, it is designed specifically for the case where the largest number of participants is limited to three. The algorithm would need to be modified and extended to handle larger numbers of participants. Secondly, some more advanced ARM instructions, such as MUL, MLA, and branch instructions, are not utilized in this algorithm due to the problem constraints. The use of these instructions might allow for further optimization and efficiency improvements in certain cases.

Despite these limitations, the algorithm is well-suited for the given problem instance and demonstrates an efficient approach to solving the MFAST problem using ARM assembly code with a limited instruction set and without loops or branches.



\section{Implementation}

The following program is an implementation of the above description. The created circuit is shown in Figure \ref{fig:Minimum_Feedback_Arc_Set_in_Tournaments}:

\begin{lstlisting}

{"register_size": 2, "run": false, "display": false}
HAD R0
HAD R1

ORACLE


; Check if R0 = 0 and R1 = 1
MOV R2, #0
MOV R3, #1
TEQ R0, R2
TEQ R1, R3
AND R4, R0, R1

; Check if R0 = 1 and R1 = 2
MOV R5, #1
MOV R6, #2
TEQ R0, R5
TEQ R1, R6
AND R7, R1, R0

; Check if R0 = 2 and R1 = 0
MOV R8, #2
TEQ R0, R8
TEQ R1, R2
AND R9, R0, R1

; Combine the results and set ZERO flag
ORR R10, R4, R7
ORR R11, R10, R9
CMP R11, #0
; ZERO flag will be set to 1 if R11 = 0, indicating a valid solution, otherwise 0



END_ORACLE

TGT ZERO

REVERSE_ORACLE

DIF {R0, R1}

STR CR0, R0
STR CR1, R1


\end{lstlisting}

\begin{figure}[htp]
    \centering
    \includegraphics[width=9cm]{Figures/Minimum_Feedback_Arc_Set_in_Tournaments_circuit.png}
    \caption{Using Grover's Algorithm to Solve the Minimum Feedback Arc Set in Tournaments Problem}
    \label{fig:Minimum_Feedback_Arc_Set_in_Tournaments}
\end{figure}

\section{Conclusion}
\label{sec:conclusion}

In this paper, we presented a novel quantum algorithm for the Minimum Feedback Arc Set in Tournaments (MFAST) problem based on Grover's Algorithm. Our approach leverages the quantum speedup provided by Grover's Algorithm to potentially achieve faster runtimes compared to classical methods for solving the MFAST problem. We provided a detailed description of the quantum circuit implementation, including the necessary oracle construction and amplitude amplification components.

Our complexity analysis demonstrated the potential for significant improvements in solving large-scale instances of the MFAST problem using our quantum algorithm. These results not only highlight the advantages of quantum computing in addressing complex optimization problems but also contribute to the ongoing development of quantum algorithms for practical applications.

Future research directions include exploring alternative oracle constructions and refining the amplitude amplification procedure for better performance. Additionally, the development of more efficient quantum algorithms for related problems, such as the general Minimum Feedback Arc Set problem, can further broaden the applicability of quantum computing in combinatorial optimization and graph theory.

